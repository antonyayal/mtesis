%poner referencias.
La seguridad de la información involucra  una serie de procesos, herramientas y métodos que al ser implementados en conjunto o individualmente mitigan el daño ocasionado por una amenaza. Poco a poco se da mayor importancia a agregar seguridad en cada una de las etapas del desarrollo de un sistema, utilizando elementos que provean soluciones efectivas y probadas. No obstante, medir qué tan seguro es un sistema es un tema controversial debido a la carencia de evaluaciones cuantitativas.
\vspace{0.3cm}

Los patrones de seguridad proporcionan una solución probada desde la fase de diseño ante un problema recurrente que coloca a un sistema en peligro de sufrir amenazas. Pero, ¿cómo evaluar la seguridad de un sistema? y ¿qué elementos son importantes para dicha evaluación? 

\vspace{0.3cm}

En este trabajo, se define un método de evaluación de la seguridad sobre un sistema informático previamente construido usando patrones de seguridad. La evaluación propuesta contempla que la seguridad, además de mitigar amenazas, también requiere de satisfacer requisitos de seguridad y políticas de seguridad. Con esto, se pretende otorgar un valor para conocer el nivel de seguridad de dicho sistema.

