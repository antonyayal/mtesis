

%----------------------------------------------------------------------
%RESUMEN
El objetivo de este capítulo es introducir los conceptos de: 1) sistemas en tiempo real; 2) tipos de ejecución de tareas; 3) el algoritmo por defecto de los sistemas en tiempo real 4) sistemas embebidos heterogéneos; 5) arquitecturas de hardware y software de tarjetas gráficas; y 6) cómputo de propósito general en en unidades de procesamiento de gráficos.
%----------------------------------------------------------------------
%SISTEMAS EN TIEMPO REAL
\section{Sistemas en tiempo real}\label{sec:sistr}

Los sistemas en tiempo real son sistemas de cómputo cuyas tareas deben actuar dentro de limitaciones de tiempo precisas ante eventos en su entorno. Por lo que el comportamiento del sistema depende, no solo del resultado del cálculo, sino también del momento (tiempo) en que se produce \cite{Buta2011}.

\subsection{Tipos de ejecución de tareas}

Existen dos tipo de ejecución de tareas, las \textit{\textbf{preemptive}}, donde es necesario interrumpir temporalmente una tarea que está realizando un sistema de cómputo, para darle la oportunidad a otra con mayor prioridad, con el compromiso de reanudar la rezagada más adelante, y las \textit{\textbf{non-preemptive} }donde se requiere que termine la tarea actual para que posteriormente inicie una con mayor prioridad.

\subsection{Algoritmos de planificación}

Earliest Deadline First (EDF) es un algoritmo óptimo de planificación para sistemas de tiempo real, y acepta tareas en modo preemptive. Es un algoritmo muy extendido en sistemas en tiempo real debido a su optimalidad teórica en el campo no-preemptive, pero al momento de implementarlo en un planificador preemptive, el resultado puede acarrear un exceso de ejecución si se toma el peor caso \cite{EmbSysDes}. Por ello es necesario buscar alternativas de algoritmos que tengan un mejor desempeño en tareas específicas.

%----------------------------------------------------------------------
%GPU
\section{GPU}

\subsection{Arquitectura Pascal}

\subsection{GPGPU}
El GPGPU (cómputo de propósito general en unidades de procesamiento de gráficos) es utilizado para acelerar el procesamiento realizado tradicionalmente por la CPU únicamente, donde la GPU actúa como un coprocesador que puede aumentar la velocidad del trabajo \cite{GpuCpu}.

%----------------------------------------------------------------------
%SISTEMAS EMBEBIDOS
\section{Sistemas embebidos}

Un sistema embebido es un sistema de cómputo diseñado para realizar tareas dedicadas, donde el mayor retos es realizar tareas específicas donde la mayoría de ellas tengan requerimientos de tiempo real \cite{LimPree}.

\subsection{Sistemas embebidos heterogéneos}
%
\vspace{0.3cm}
En los últimos años los sistemas embebidos han ido demandando nuevas características debido a su rápida adopción en el mercado. Con lo que surge el desarrollo de sistemas embebidos heterogéneos, donde está contemplado realizar una gran cantidad de cómputo pero con una gran eficiencia tanto energética como en espacio.
\vspace{0.3cm}

Actualmente la empresa NVIDIA tiene en su catálogo sistemas embebidos heterogéneos con un gran soporte y bibliotecas para el cómputo de alto rendimiento. Dichos sistemas cuentan con la arquitectura pascal de última generación \cite{GPUArt}, la cual permite compartir memoria entre CPU y GPU.

\subsubsection{Jetson TX2}

% Framework
Debido a que la mayoría de las GPU en sistemas embebidos no son de naturaleza preemptive, es importante programar los recursos de GPU de manera eficiente en múltiples tareas \cite{TX2I} ya sea de planificación o memoria, lo que permite pensar en un framework que ayude a la administración de sus características. 

\section{Resumen}

En este capítulo se presenta una breve introducción a la seguridad de la información, la importancia de incluirla en los sistemas de software y las amenazas a las que están expuestos los sistemas. Se hace un énfasis en la inclusión de la seguridad en la etapa de diseño de un sistema, donde se explica que utilizar guías para proporcionar un nivel de seguridad a un sistema en diseño disminuye las posibilidades de una amenaza al sistema ya implementado. 


