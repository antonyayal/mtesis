\documentclass[letter,10pt]{article}

\usepackage{tikz,pgflibraryshapes}
\usetikzlibrary{arrows, calc, decorations.pathmorphing}
\usepackage[psfixbb,graphics,tightpage,active]{preview}


\PreviewEnvironment{tikzpicture}
\newlength{\imagewidth}
\newlength{\imagescale}

\usepackage{xcolor}
\usepackage{color}


\definecolor{orange}{RGB}{255,127,0}
\definecolor{gris}{RGB}{230,230,230}

\usetikzlibrary{shapes,arrows}
\pgfdeclarelayer{background}
\pgfdeclarelayer{foreground}
\pgfsetlayers{background,main,foreground}
\tikzstyle{startstop} = [rectangle, rounded corners, minimum width=3cm, minimum height=1.5cm,text centered, draw=black, fill=white!30,double,text width=2.5cm]

\tikzstyle{objective} = [rectangle, rounded corners, minimum width=3cm, minimum height=1.5cm,text centered, draw=black, fill=white!30,text width=2.5cm]

\tikzstyle{system} = [rectangle, minimum width=3cm, minimum height=1.5cm,text centered, draw=black, fill=white!30,text width=2.5cm]

\tikzstyle{pattern} = [rectangle, rounded corners, minimum width=2cm, minimum height=0.6cm,text centered, draw=black, fill=white!30]

\tikzstyle{arrow} = [thick,->,>=stealth]
\tikzstyle{darrow} = [thick,<->,>=stealth]

\begin{document}

\begin{tikzpicture}[node distance=1cm]

	
	\node (sreq) [startstop] {Requisitos de seguridad};

	
	%Bloque: Patron de seguridad
	\node (pattern) [objective,left of = sreq,xshift=-4cm,double] {Patrones de seguridad};
	\node (sistema) [draw=none,fill=none, above of = pattern,yshift=0.5cm] {Modelo del sistema};
	
	%Bloque: Metodo de evaluacion	

	\node (eval) [system,yshift=-2cm, below of = sreq,text width=2.5cm] {Evaluaci\'on de seguridad del sistema};
	\node (resul) [objective,yshift=-1cm, below of = eval,text width=3cm,double,yshift=-1cm] {Resultado de la evaluaci\'on};
	\node[draw=none,fill=none, below of= resul,yshift=-0.5cm] {M\'etodo de evaluaci\'on};
	
	
	%Bloque: Modelo de amenazas
	
	\node (att1) [pattern,xshift=7cm,yshift=-2,left of = eval] {Amenaza$_{11}$};
	\node (susp1) [draw=none,fill=none,below of=att1,] {\vdots};
	\node (att2) [pattern, below of = susp1] {Amenaza$_{1n}$};
	\node (modAme) [draw=none,fill=none, above of = att1] {Amenazas $_1$};
	\node (vacio3)  [draw=none,fill=none,left of = att1,xshift=-0.5cm] {};
	\node (uc) [startstop,above of = att1,yshift=2cm] {Caso de Uso$_1$};
	
	\node (att3) [pattern,xshift=4cm,right of = att1] {Amenaza$_{N1}$};
	\node (susp2) [draw=none,fill=none,below of=att3] {\vdots};
	\node (att4) [pattern, below of = susp2] {Amenaza$_{Nn}$};
	\node (modAme1) [draw=none,fill=none, above of = att3] {Amenazas $_N$};
	\node (puntos)  [draw=none,fill=none,right of = uc,xshift=1.5cm] {...};
	\node (puntos1)  [draw=none,fill=none,right of = susp1,xshift=1.5cm] {...};
	\node (uc1) [startstop,above of = att3,yshift=2cm] {Caso de Uso$_N$};

	
	%Flechas del diagrama
	\draw [arrow] (uc) -- (modAme);
	\draw [arrow] (uc1) -- (modAme1);
	\draw [arrow] (vacio3) -- (eval);
	\draw [arrow] (pattern) |- (eval);
	\draw [arrow] (sreq) -- (eval);
	\draw [arrow] (eval) -- (resul);


	\begin{pgfonlayer}{background}
	
		\path (modAme.west |- modAme.north)+(-0.6,0) node (j) {};
		\path (att4.east |- att4.south)+(+0.6,-0.9) node (k) {};
            	\path[fill=white!20,rounded corners, draw=black!50, dotted,thick] (j) rectangle (k);  
	
		\path (att1.west |- att1.north)+(-0.3,0.3) node (a) {};
		\path (att2.east |- att2.south)+(+0.3,-0.5) node (b) {};
            	\path[fill=white!20,rounded corners, draw=black!50, dotted,thick] (a) rectangle (b);  
	
		\path (att3.west |- att3.north)+(-0.3,0.3) node (c) {};
		\path (att4.east |- att4.south)+(+0.3,-0.5) node (d) {};
            	\path[fill=white!20,rounded corners, draw=black!50, dotted,thick] (c) rectangle (d);  
	
		\path (eval.west |- eval.north)+(-0.5,0.3) node (e) {};
		\path (resul.east |- resul.south)+(+0.5,-0.5) node (f) {};
            	\path[fill=white!20,rounded corners, draw=black!50, dashdotted] (e) rectangle (f);  
	
		\path (pattern.west |- pattern.north)+(-0.5,0.3) node (h) {};
		\path (uc1.east |- uc1.south)+(+0.5,-0.3) node (i) {};
            	\path[fill=gris!20, draw=black!50] (h) rectangle (i);  
	
		
		
    	\end{pgfonlayer}
\end{tikzpicture}



\end{document}