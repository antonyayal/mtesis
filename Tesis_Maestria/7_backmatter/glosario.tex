%Glosario

\newglossaryentry{preemptive}
{
  name={preemptive},
  description={Modo en que una tarea es interrumpida temporalmente por un planificador.}
}

\newglossaryentry{non-preemptive}
{
  name={non-preemptive},
  description={Ver cooperativo.}
}

\newglossaryentry{cooperativo}
{
  name={cooperativo},
  description={Modo en que las tareas se ejecutan ininterrumpidamente hasta que terminan su procesamiento.}
}
\newglossaryentry{cooperativa}
{
  name={cooperativa},
  description={Modo en que las tareas se ejecutan ininterrumpidamente hasta que terminan su procesamiento.}
}

\newglossaryentry{preemption}
{
  name={preemption},
  description={Hecho de ser preemptive.}
}


\newglossaryentry{Throughput}
{
  name={throughput},
  description={Tasa de transferencia.}
}

\newglossaryentry{framework}
{
  name={Framework},
  description={Entorno de trabajo que define una estructura en concreto para facilitar las tareas.}
}

\newglossaryentry{pixel}
{
  name={pixel},
  description={Picture Element o Elemento de Imagen) Unidad mínima homogénea de color que forma una imagen.}
}

\newglossaryentry{texel}
{
  name={pixel},
  description={Texture Element o Elemento de Textura)  Unidad mínima de una textura aplicada a una superficie.}
}

\newglossaryentry{kernel}
{
  name={kernel},
  description={Función o fragmento de código acelerado en una GPU. No está relacionado con el kernel de un Sistema Operativo.}
}

\newglossaryentry{tcpip}
{
  name={Protocolo de Control de Transmisión/Protocolo de Internet},
  description={Es un modelo de referencia que se basa en una pila de protocolos independientes distribuidos en cuatro capas, la de aplicación, transporte, interred y enlace. \cite{tanenbaum_wetherall_2012}}
}
\newglossaryentry{deadline}
{
  name={deadline},
  description={Plazo límite, es el momento justo antes en que una tarea debe completar su ejecución.}
}
\newglossaryentry{quantum}
{
  name={quantum},
  description={Intervalo de tiempo durante el cual se permite que un proceso se ejecute en un sistema multitarea preemptive.}
}
\newglossaryentry{DMA}
{
  name={Acceso directo a memoria},
  description={Permite a ciertos dispositivos de diferentes velocidades acceder a la memoria del sistema para leerla o escribirla sin pasar por el CPU, esto sin generar una carga masiva de interrupciones.}
}
\newglossaryentry{barrera}
{
  name={barrera},
  description={Un método de sincronización. Cuando en el código fuente se encuentra una barrera, el grupo de procesos debe detenerse hasta que todos ellos lleguen a ella.}
}

\newglossaryentry{Slurm}
{
  name={Slurm},
  description={Simple Linux Utility for Resources Management, es un sistema de gestión de tareas y de clusters\cite{Slurm}.}
}

\newglossaryentry{proxy}
{
  name={proxy},
  description={Es un agente o representante, ya sea dispositivo o programa, que está autorizado para actuar en nombre de otra instancia.}
}
\newglossaryentry{daemon}
{
  name={daemon},
  description={Es un tipo especial de proceso informático no interactivo, es decir, que se ejecuta en segundo plano en vez de ser controlado directamente por el usuario.}
}

\newglossaryentry{contexto}
{
  name={contexto},
  description={Las estructuras, instancias u objetos que contienen un conjunto mínimo de atributos, propiedades o estados que permiten ejecutar o administrar un conjunto definido de tareas.}
}

\newglossaryentry{threshold}
{
  name={threshold},
  description={Límite que marca hasta que punto cambia algo.}
}

\newglossaryentry{triggers}
{
  name={triggers},
  description= {También llamado disparadores, son elementos que desencadenan otros procesos.}
}

\newglossaryentry{throughput}
{
  name={throughput},
  description={La tasa de transferencia efectiva. Es el volumen de trabajo o de información neto que fluye a través de un sistema.}
}


%Acronimos
\newacronym{DRAM}{DRAM}{Memoria dinámica de acceso aleatorio}
\newacronym{DMA}{DMA}{Acceso directo a memoria}
\newacronym{Slurm}{Slurm}{Simple Linux Utility for Resources Management}
\newacronym{TB}{TB}{Threads por Bloque}
\newacronym{CPU}{CPU}{Unidad de Procesamiento Central}
\newacronym{GPU}{GPU}{Unidad de Procesamiento de Gráficos}
\newacronym{GPGPU}{GPGPU}{Cómputo General de Unidades de Procesamiento de Gráficos}