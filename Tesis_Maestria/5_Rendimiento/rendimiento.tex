\chapter{Rendimiento}
    \label{cha:Rendimiento}

Este capítulo propone posibles métricas para evaluar el rendimiento del framework. Para poder realizar cualquier evaluación del rendimiento es necesario obtener datos importantes sobre las ejecuciones de un kernel, con dicha información se podrá implementar una serie de gráficas que permitan valorar la tendencia de los resultados.
\newline

Como ejemplo tomaremos a Earliest Deadline First (algoritmo \ref{alg:edf}), un algoritmo de planificación dinámica que selecciona un conjunto de tareas \textit{$R=\{\tau_1,\tau_2,...,\tau_n\}$} de acuerdo a sus plazos absolutos \textit{d}. Las tareas con plazos más próximos se ejecutarán con una prioridad más alta\cite{Buta2011}. 
Debido a que el plazo límite absoluto de una tarea periódica \textit{$\tau_i$} depende de la actual instancia \textit{$\tau_j$}, plazo límite relativo \textit{D}, un periodo de activación \textit{T} y una fase o tiempo de activación de la primera instancia \textit{$\Phi$}, tenemos que:

\begin{equation}
d_{i,j} = \Phi_i + (j-1)T_i + D_i
\end{equation}

EDF no realiza ninguna suposición sobre si las tareas son periódicas o aperiódicas debido a que realiza una asignación dinámica. 

\lstinputlisting[style=CStyle, frame=single,label=alg:edf,  basicstyle=\ttfamily\footnotesize, caption=Algoritmo de planificación EDF.]{algorithms/alg_edf.c}

Al ejecutarse normalmente en escenarios preemptive, la tarea que se está ejecutando actualmente realiza una suspensión preemptive y se da lugar a cualquier otra que posea una fecha límite más próxima. Por esta flexibilidad es el algoritmo más extendido a la hora de pensar en implementar un planificador.
\newline

Para conocer si un conjunto de tareas es planificable se utiliza la función de utilidad \textit{U}, la cual describe el consumo de cómputo \textit{$C_i$} de las tareas entre su periodo \textit{$T_i$}. Para evitar perder los tiempos límite debemos mantenerla inferior a 1.

\begin{equation}
U=\sum_{i=1}^{n} \frac{C_i}{T_i} \leq 1
\end{equation}

Para evitar que una tarea pierda su plazo límite es necesario prevenir que esté bloqueada por más de \textit{$D_i - C_i$} unidades de tiempo, asegurando que a regreso a línea no se presente el plazo vencido $dm$.

%El conocer la cantidad de plazos vencidos \textit{$n_{dm}$} es importante para evitar sobrecargar el sistema y darle la oportunidad a todas las tareas de consumir recursos eventualmente. 
%Es necesaria una métrica que nos permita visualizar el número de plazos vencidos \textit{$\tau_{i[dlmss]}$} debido al tiempo invertido en los cambios de contexto, es decir, de aquellas tareas que estuvieron en backup y que regresaron a linea, pero con plazos ya vencidos.
La métrica que nos permite evaluar el rendimiento del sistema en cuanto al número de plazos vencidos \textit{$n_{dm}$} es:
\begin{equation}
n_{dm}=\sum_{i=1}^{n} dm_{i}
\end{equation}

Una tarea tiene un número total de subkernels \textit{$n_{sk}$}, y \textit{$n_{bk}$} cambios de contexto, donde  \textbf{$n_{bk} = n_{sk}-1$}.

Al sumar el tiempo total de inserciones en la estructura backup \textit{$t_{bk[in]}$} con el de las extracciones \textit{$t_{bk[ex]}$} obtenemos el tiempo total del cambio del contexto \textit{$t_{bk}$}.

\begin{equation}
t_{bk}=t_{bk[in]}+t_{bk[ex]}
\end{equation}

El tiempo de ejecución total con suspensión preemptive \textit{$t_p$} nos indica el tiempo total que utiliza el programa tomando en cuenta el tiempo que se utiliza en los cambios de contexto. 
En ocasiones es importante conocer el tiempo total del cálculo \textit{$t_{np}$} de cada tarea en el ambiente preemptive, para comparar la variación \textit{$S(p)$} que pudiera existir con el tiempo original \textit{$t_{or}$} en un ambiente sin planificador, donde \textit{$S(p) > 1$} significa una mejora en la ejecución de la tarea.

\begin{equation}
S(p)=\frac{t_{or}}{t_{np}}
\end{equation}

El tiempo fuera de línea o de ociosidad \textit{$t_{idle}$} indica el tiempo en que un kernel está fuera de operación, y es necesario mantenerlo en un nivel bajo, para que se garantice el eventual consumo de recursos, y se eliminen los posibles plazos vencidos.
\newline