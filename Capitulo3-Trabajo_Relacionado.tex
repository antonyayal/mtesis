%%Capítulo 3 trabajo relacionado 

%----------------------------------------------------------------------
    %RESUMEN
Este capítulo presenta los trabajos relacionados con el tema de esta tesis, se analizan 
\begin{inparaenum}
	\item Planificación de EDF preemptive limitado de sistemas con tareas esporádicas
	 (\textit{Limited Preemption EDF Scheduling of Sporadic Task Systems});
	 %
	 \item Planificación de recursos espaciales compartidos con prioridad para unidades de gráficos embebidos 
	 (\textit{Priority-driven spatial resource sharing scheduling for embedded graphics processing units});
    	 %
	\item Framework para planificación en tiempo de ejecución de aplicaciones con manejo de eventos en sistemas embebidos basados en GPU 
	(\textit{Run-Time Scheduling Framework for Event-Driven Applications on a GPU-Based Embedded System});
	%
    	\item Planificación conjunta con GPU y aseguramiento de la memoria intra-nodo 
	(\textit{Intra-Node Memory Safe GPU Co-Scheduling});
	%
	\item Sobre planificación dinámica para la GPU y sus aplicaciones en computación gráfica y más
	(\textit{On Dynamic Scheduling for the GPU and its Applications in Computer Graphics and Beyond}); y
	%
	\item REGM: Un modelo de ejecución GPGPU responsivo para soluciones en tiempo de ejecución
	(\textit{REGM: A Responsive GPGPU Execution Model for Runtime Engines});
 \end{inparaenum}
%----------------------------------------------------------------------

\vspace{0.3cm}
Cada sección presenta lo propuesto en el trabajo relacionado, donde se describe el problema, los objetivos y la solución a éste. Brevemente se describe la solución propuesta con los resultados obtenidos y por último se presentan las conclusiones del trabajo.

%Evaluating the degree of security of a system built using security patterns
%\section{Evaluar el grado de seguridad de un sistema construido usando patrones de seguridad}
%\subsection{Introducción}

\section{Resumen}

%En este capítulo se presenta el resumen de tres trabajos relacionados con la evaluación de los patrones de seguridad. El primer trabajo presenta una métrica de seguridad denominada SC la cual contabiliza el total de amenazas mitigadas por patrones de seguridad entre el total de amenazas. Una de las mejoras que propone es utilizar la aproximación \textit{Twin peaks} que produce una nueva arquitectura en cada ciclo contemplando los mismos casos de uso pero a mayor detalle.

%\vspace{0.3cm}

%El segundo trabajo presenta una metodología que consiste en medir qué extensión de una arquitectura está protegida con respecto a las amenazas de seguridad más relevantes. La metodología consiste en cuatro partes: 1) mapeo de las amenazas con los objetivos de seguridad, 2) clasificación de las amenazas de acuerdo a su severidad, 3) determinación de la protección ante una amenaza y 4) cálculo de la cobertura de seguridad. 

%\vspace{0.3cm}

%Por último, el tercer trabajo presenta una metodología que permite elegir los patrones de seguridad con respecto a los objetivos de seguridad y las métricas que evaluarán a los patrones. La metodología se divide en tres fases que son: 1) definición de los patrones de seguridad a partir de los objetivos de seguridad, 2) selección de métricas e 3) interpretación de resultados. Este trabajo tiene como objetivo integrar las métricas a la evaluación de un sistema que está utilizando los patrones de seguridad. 