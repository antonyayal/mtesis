%%Latín
% e.g.    -> por ejemplo (analogias o ejemplos para generar contexto)
% et al. -> y otros (para autores)
% i.e.    -> es decir  (explicación más detallada)

\section{Contexto}

AAAAAAaAAAAAA

La información es un activo estratégico para las empresas. La existencia de vulnerabilidades en los sistemas que comprometan la información pone en riesgo el éxito de la empresa. La seguridad de la información se enfoca en preservar la confidencialidad, integridad y disponibilidad a los datos de un sistema. Debido a la importancia de la información, se crea la rama de la tecnología denominada seguridad informática, encargada de hacer que se cumplan los principios de la seguridad de la información, minimizando los riesgos físicos o lógicos a los que esté expuesto el sistema \cite{Var14, Wyl03,Urb16}.

\vspace{0.3cm}

El área de seguridad de la información se considera inmadura. Uno de los aspectos en los que falta profundizar son  los problemas de seguridad  asociados al desarrollo de un sistema. Como consecuencia se carece de evaluaciones objetivas donde se indique qué tan seguros son los sistemas desarrollados. 

\vspace{0.3cm}

Investigaciones recientes se enfocan en generar evaluaciones que indiquen cuán seguro es el sistema que se está desarrollando y corregir posibles vulnerabilidades durante las etapas del ciclo de vida de software. Los patrones de seguridad son una herramienta para diseñar sistemas más seguros \cite{OrtGarFer11,Var1103, JahWMe10}.

\vspace{0.3cm}

Contar con evaluaciones en seguridad de la información ayuda a la toma de decisiones relacionadas con dicho activo, ya que el resultado de una evaluación revela la condición de un sistema o la magnitud de un fenómeno ocurrido, lo que permite tomar alguna acción. Entre las razones por las cuales evaluar la seguridad de la información es importante se encuentra principalmente la económica, debido a que se estima una pérdida de entre el 1\% al 5\% la empresa posterior a un ciberataque \cite{AtzLio0605,Var1103,BroHin1304}. 

\vspace{0.3cm}

Además de las cuestiones económicas que conlleva medir la seguridad de la información, existe la parte tecnológica en el desarrollo de los sistemas. Como dijo Lord Kelvin \textit{``Si no puedes medirlo, no podrás mejorarlo''}. La existencia de evaluaciones en esta área también contribuye mejorar las tecnologías con las que se desarrollan. Tener una evaluación que indique cuán seguro es el sistema apoya a que los investigadores y desarrolladores de la tecnología mejoren sus productos.

\vspace{0.3cm}

Específicamente durante la etapa de diseño intervienen los patrones de seguridad, los cuales describen una solución en forma de guías y reglas sobre un problema de seguridad que está asociado a un activo. Existe una gran variedad de patrones de seguridad como la colección mostrada en \cite{SchFerHyb06}.  

\vspace{0.3cm}

Se sabe que la seguridad es un tema subjetivo, tornando complejo querer evaluarlo. No obstante, esta complejidad no ha sido obstáculo para que exista una amplia variedad de estudios enfocados a mejorar el conocimiento que se tiene sobre este tema y de cómo estructurar una evaluación objetiva. Para abordar el problema, primero se debe formalizar el objetivo a alcanzar y las propiedades del sistema. Posteriormente, se procede a utilizar herramientas formales y automáticas para evaluar la seguridad (las herramientas para evaluar la seguridad deben extrapolarse a cualquier sistema) \cite{AtzLio0605}. 

\section{Problema}

El problema abordado en esta tesis es evaluar la seguridad de un sistema ya creado, específicamente los sistemas que desde el diseño fueron construidos utilizando patrones de seguridad como un conjunto. 

\vspace{0.3cm}

La evaluación debe proporcionar un parámetro que ayude a los diseñadores y desarrolladores a mejorar los productos de software a los que se quiere proporcionar seguridad. 

\section{Hipótesis}

La hipótesis del presente trabajo es:

\begin{quote}
	\textit{Se puede evaluar la seguridad de un sistema de forma sistemática de tal manera que se proporcione una métrica la cual indique bajo cierto criterio si es seguro o no, si previamente se sabe que ha sido construido usando patrones de seguridad.}
\end{quote}


\section{Aproximación}

Mediante el análisis de los elementos inherentes a un sistema como los diagramas UML, requisitos de seguridad y políticas de seguridad, el presente trabajo presenta un método para evaluar la seguridad de los sistemas de información que previamente han sido construidos con patrones de seguridad. Se realiza la evaluación de un sistema que consiste en aplicar el método presentado y se proporciona un valor que indica cuán seguro puede ser considerado. 

\section{Contribuciones}

El objetivo principal del presente trabajo es proporcionar una evaluación que contribuya a definir un nivel de seguridad de un sistema que utiliza patrones de seguridad y considerar que los requisitos y políticas de seguridad también son una parte importante de la evaluación de seguridad de cualquier sistema. El análisis de un sistema utilizando esta evaluación apoyaría a  diseñadores y desarrolladores en conocer cuál es la cobertura de amenazas, requisitos de seguridad y políticas de seguridad del sistema para aplicar acciones de ser necesario. 


\section {Estructura de la tesis}

El presente trabajo se estructura en capítulos, los cuales se describen brevemente a continuación:

\begin{enumerate}[label=Capítulo \arabic*,leftmargin=*]
	\setcounter{enumi}{1}
	\item \textbf{Antecedentes}
	
		\vspace{0.3cm}
	
		Aquí se presentan los conceptos básicos necesarios para entender el objetivo de la tesis y sus contribuciones, explicando de manera detallada por qué la hipótesis presentada. Primero, se describe la importancia de proteger la información manipulada por un sistema informático. Después, se describe cómo se encuentra inmersa la seguridad en el diseño de un sistema, donde se explica cómo los patrones de seguridad ayudan a prevenir ataques conocidos. Se da una descripción breve de los patrones de seguridad y finalmente, se muestran las mediciones relacionadas con la seguridad de los sistemas y las mediciones de seguridad sobre sistemas.
	\item \textbf{Trabajo relacionado}
	
	\vspace{0.3cm}
	
		En este capítulo, se describe el trabajo relacionado a las métricas asociadas con sistemas que usan patrones de seguridad. Primero, se presenta el artículo titulado ``\textit{Measuring the level of security introduced by security pattern}''. En el cual presenta una metodología para comparar dos sistemas sobre el nivel de seguridad que les otorgan los patrones de seguridad al ser aplicados. Posteriormente, se presenta el artículo titulado ``\textit{Towards a quantitative assessment of security in software architectures}'', donde el principal objetivo es proporcionar un valor cuantitativo sobre el nivel de seguridad de un sistema mediante el uso de árboles de requisitos. Finalmente, se presenta el artículo titulado ``\textit{Using security patterns to combine security metrics}''. En este trabajo, se enfocan en seleccionar las métricas correctas relacionadas a los patrones de seguridad y cómo interpretar sus resultados. 
		  
		\item \textbf{Evaluación de seguridad al implementar patrones de seguridad}
		
		En esta parte de la tesis se presenta de forma general los elementos necesarios para la evaluación propuesta, así como las características que se deben considerar y cómo se debe manipular la información previa requerida. De manera más detallada, en las subsecciones se presenta la propuesta de evaluación y cómo interpretar los resultados obtenidos.
		\item \textbf{Caso de estudio del método propuesto}
		
		En este capítulo se utiliza como ejemplo un sistema financiero básico sobre el cual se aplica el método planteado en el Capítulo 4.
		\item \textbf{Conclusiones}
		
		Luego de presentar el método de evaluación, se discuten y analizan los resultados obtenidos en la tesis. Además, se proponen trabajos futuros que den continuidad al trabajo presentado.
		
\end{enumerate}
